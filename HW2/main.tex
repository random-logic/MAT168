\documentclass[17pt]{extarticle}

% Language setting
% Replace `english' with e.g. `spanish' to change the document language
\usepackage[english]{babel}

% Set page size and margins
% Replace `letterpaper' with `a4paper' for UK/EU standard size
\usepackage[letterpaper,top=2cm,bottom=2cm,left=3cm,right=3cm,marginparwidth=1.75cm]{geometry}

% Useful packages
\usepackage{amsmath}
\usepackage{graphicx}
\usepackage[colorlinks=true, allcolors=blue]{hyperref}
\usepackage{enumitem}

\setlength{\parindent}{0cm}

\title{MAT168 HW2}
\author{Andrew Jowe}

\begin{document}
\maketitle
\section*{(1)(2)(3)}
Please refer to written part.

\section*{(4a)}
Let $x_j$ denote the quantity of $j-th$ food consumed.

\bigskip The primal problem is:
$\\min \ c^Tx$
$\\Ax \geq b$

\section*{(4b)}
Let $y_i$ denote the quantity of $i-th$ nutrient consumed.

The dual problem is:
$\\max \ b^Ty$
$\\A^Ty \leq c$

\section*{(4c)}
A few examples:

\bigskip A homeless person only has a couple dollars to spare on any food for the day. Maximizing nutrition is very important to prevent serious health issues. That person needs to make sure that they spend wisely to maximize health. Solve the problem so that the person can maximize nutrition within the tight budget.

\bigskip Andrew just learned about the roles of the nutrients in his nutrition class, and thus wants to ensure that his nutrition is maximized so that he can stay healthy. As a college student with no financial aid, tuition is too darn high which leaves Andrew with a very tiny budget for healthy foods. Solve the problem so that Andrew can maximize nutrition without breaking the bank.

\bigskip Anya has poor eating habbits, she eats too much junk food and not enough healthy food, and thus was told by the doctor that she has to maximize nutrition from now on before she gets health related issues. Since Anya is not super rich, she cannot overspend. Solve the problem so that Anya can maximize nutrition without getting ripped off.

\section*{(4d)}
Here are a couple of ways where reality proves our model wrong.

\bigskip For both models, one of the real world factors that isn't taken account for is the diminising returns for nutrients. While it is crucial to consume enough nutrients, too much nutrients doens't realistically benefit your body either. For example, too little body fat can make a person boney which isn't good. On the other hand, too much body fat can make a person obese which also isn't good. This model just assumes the more the better. A good way to fix this issue is to set another constraint such that the amount of fat cannot exceed a certain amount.

\bigskip For the dual model, another real world factor that isn't taken account for is that nutrients cannot be substituted. Suppose you are currently intaking the optimal amount of nutrients and you happen to be consuming exactly the recommended daily value (RDA) of all nutrients. For example, suppose people start to realize that they want to be healthier which increases demand for healthy foods and decreases demand for junk foods. When healthy foods become more expensive and junk foods become cheaper, you may get a higher optimal value by decreasing healthy foods and increasing junk foods. But, in reality, this actually makes you less healthy since junk foods usually contain more fat and less of other nutrients compared to healthy foods. If you follow this model, not only would you be consuming too much fat which can lead to obesity, you would also have a deficiency in all other nutrients based on the RDA which would also lead to serious health problems. In this example, you will become a lot less healthy if you follow the model. A good way to fix this issue is to encourage people not to use this model unless they are super poor and cannot afford to consume the RDA regardless of how they spend their money.

\section*{Collaboration}
All collaborators are listed (in alphabetical order) below:
\begin{itemize}
    \item Andrew
    \item Zhongning
    \item Sterling
    \item Fenqgin
    \item Anne
    \item Dhruv
    \item Alisa
\end{itemize}

\section*{Academic Integrity}
On my personal integrity as a student and member of the UCD community, I have not given, nor received and unauthorized assistance on this assignment.

Signature: Andrew Jowe
\end{document}